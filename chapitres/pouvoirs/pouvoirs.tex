\section{La Force}
\label{sec:force}

Abordons le chapitre de La Force. Si vous souhaitez jouer un Jedi ou un Sith c’est ici que cela se passe. Mais d’abord voyez avec votre MJ s’il est d’accord.

Avant de rentrer dans le vif du sujet, il est nécessaire de donner quelques avertissements. La création d’un Jedi est volontairement difficile et demande une bonne dose d’expérience. En premier lieu parce que les Jedi et les Sith ne constituent pas une population très répandue dans la galaxie. En second lieu, parce qu’une fois que vous êtes Jedi ou Sith, il n’y a plus grand chose qui va pouvoir vous résister en dehors d’autres Jedi et Sith. Du coup le fun en prend un coup, les variantes d’adversaires chutant drastiquement.

Ceci dit avoir un Jedi dans l’équipe peut aussi ouvrir des horizons nouveaux au MJ.

\subsection{Créer un Jedi/Sith}
La première chose pour créer un héros sensible à la force est de prendre l’atout \emph{Arcane (Force)} qui représente la faculté de votre héro à percevoir La Force. Cet atout ne peut être pris qu’à la création du personnage.

Ensuite vous devez investir des points dans la compétence \emph{Maîtrise de la Force} qui représente votre niveau de sensibilité à La Force et votre maîtrise de cette dernière.

Les héros utilisant la Force le font en consommant des Points de Pouvoir. Vous recevez \textbf{10} PP à la création de votre personnage. L’utilisation de certains pouvoir (actif) nécessite de dépenser ces points de pouvoir. Un héros gagne un point de pouvoir par heure.

Les utilisateurs de la Force débutent avec \textbf{2} pouvoirs au choix parmi les pouvoirs Innés. Un utilisateur de la Force peut apprendre un nouveau pouvoir en choisissant l’Atout \emph{Nouveau pouvoir}. Dès qu’il choisit cet Atout lors d’une Progression, il peut immédiatement utiliser le nouveau pouvoir sélectionné. Attention, certains pouvoirs nécessitent d’avoir l’Atout \emph{Padawan} ou \emph{Apprenti Sith} et donc d’être un Jedi ou un Sith.

\subsection{Le serment}
\label{sec:force-serment}
Pour qu’un héros puisse prétendre au titre de \emph{Maître} il doit prêter serment à son ordre (ou à son maître pour les Sith). Cela implique qu’il devrait respecter le code de son ordre sous peine de succomber à la Force.

Le personnage ayant le Handicap \emph{Serment} se doit de respecter le code de son penchant à la Force. Pour chaque action qu’un héros effectue à l’encontre de son serment, il prend 1 point de Corruption. 

Un héros corrompu doit faire un jet d’\^Ame avant de pouvoir utiliser la Force. Pour chaque niveau de Corruption, le malus de ce jet augmente de -2. Il est possible de perdre un point de Corruption en sacrifiant une progression. Au troisième point de Corruption, deux solutions sont possibles (selon le MJ):

\begin{description}[align=left] 
    \item [Perte de contrôle]
        Le héros succombe à la Force et perd tout contrôle, il devient alors un PNJ. Le joueur derrière le héros à fait trop d’erreur et perd son héros.

    \item [Changement de coté]
    	Le héros change de coté de la Force. Il perd son atout de \emph{Maître Jedi/Sith} et son Handicap de \emph{Serment}. Il perd aussi tout ses pouvoirs non élémentaires. Ce changement coûte une progression.
\end{description}

\subsubsection{Le code Jedi}
\begin{quotebox}
Il n’y a pas d’émotion, il y a la paix. \\
Il n’y a pas d’ignorance, il y a la connaissance. \\ 
Il n’y a pas de passion, il y a la sérénité. \\ 
Il n’y a pas de chaos, il y a l’harmonie. \\ 
Il n’y a pas la mort, il y a la Force. 
\end{quotebox}

Le Jedi pour respecter son serment ne doit jamais invoquer la Force sous le coup d’émotions, il ne peut entreprendre d’action pour son propre profit et ne doit pas nuire à son prochain. C’est au MJ de juger que le joueur respecte le roleplay de son personnage.

\subsubsection{Le code Sith}
\begin{quotebox}
La paix est un mensonge, il n’y a que la passion. \\
Par la passion, j’ai la puissance. \\
Par la puissance, j’ai le pouvoir. \\
Par le pouvoir, j’ai la victoire. \\
Par la victoire, je brise mes chaînes. \\
La Force me libérera.
\end{quotebox}

Le Sith doit laisser libre court à ses émotions et n’utilise la force que pour son profit. C’est au MJ de juger que le joueur respecte le roleplay de son personnage.

\begin{center}
	\vspace*{\fill}
	\includegraphics[width=0.7\linewidth]{img/pouvoirs/jedi-sith.png}
	\vspace*{\fill}
\end{center}

\clearpage
\subsection{Les pouvoirs}

\subsubsection{Pouvoirs Innés}
Il s’agit des pouvoirs qui peuvent être développés sans que le héros n’ait suivi d’apprentissage. Le plus souvent développés de façon innée, ils sont un indice pour les Jedi ou les Sith quand à la sensibilité d’un individu à la Force.

\begin{description}[align=left] 
    \item [Vision de Force] ~ \\

        \begin{tabular}{ r l }
            \textbf{Rang}    & Novice \\
            \textbf{Points}  & 2 \\
            \textbf{Portée}  & Visuelle \\
            \textbf{Durée}   & 1pt / 10mn \\
        \end{tabular}
        \\ \\
        La vision de Force est une capacité assez simple qui peut être assimilée d’une façon plus générale au sens de Force. Toutefois, il faut faire la part des choses entre la vision de Force et le sens de Force. En effet, le sens de Force ne permet à son utilisateur que d’obtenir une perception assez générale de son environnement, alors que la vision de Force permet d’augmenter l’acuité de ce que voit son utilisateur, peu importe l’environnement dans lequel il se trouve. De même, en se focalisant sur une sensation précise, l’utilisateur de la vision de Force peut même voir à travers les murs.
        \\

    \item [Sens de Force] ~ \\

        \begin{tabular}{ r l }
            \textbf{Rang}    & Novice \\
            \textbf{Points}  & 2 \\
            \textbf{Portée}  & Visuelle ou Infinie \\
            \textbf{Durée}   & 3 (1 / round) ou 1 heure (1 / heure) \\
        \end{tabular}
        \\ \\
        Le pouvoir appelé sens de Force permet à un utilisateur de ressentir les émotions et les sentiments d’autres êtres vivants à travers la Force. Mais ce pouvoir ne se limite pas à cette simple perception ; cette technique est tout aussi pratique pour décrypter un avenir toujours changeant. Le sens de Force est aussi un moyen de ressentir les grandes secousses pouvant se produire dans la Force, comme la destruction d’une planète, entraînant la mort de millions d’individus. 

        Enfin, selon le niveau d’Arcane(Force) du joueur il sera possible via un jet de Perception supplémentaire de déceler la présence de la Force dans un lieu ou sur un individu.

        Un individu non formé, qui utilise ce pouvoir de façon innée ressent ce pouvoir comme de fortes intuitions.
        \\  

    \newpage
    \item [Saut de Force] ~ \\

        \begin{tabular}{ r l }
            \textbf{Rang}    & Novice \\
            \textbf{Points}  & 2 \\
            \textbf{Portée}  & Personnelle \\
            \textbf{Durée}   & Instantané \\
        \end{tabular}
        \\ \\
        Bien qu’étant considéré comme un pouvoir de base, le saut de Force n’en est pas moins très utile pour de nombreux pratiquants. En effet, il permet à un utilisateur de se dégager d’une situation difficile, comme un encerclement ennemi, ou bien d’atteindre d’un unique bond un endroit surélevé ou isolé. Son utilisation est fort simple par ailleurs, puisque le Jedi ne fait qu’amplifier ses aptitudes naturelles à sauter grâce à la Force. 

        La qualité du saut, que ce soit en hauteur ou en longueur, dépend évidemment de la maîtrise qu’a l’individu de la Force. Le saut atteint 10m (4 cases) et double avec une relance.
        \\

    \item [Vitesse de Force] ~ \\

        \begin{tabular}{ r l }
            \textbf{Rang}    & Aguerri \\
            \textbf{Points}  & 1 \\
            \textbf{Portée}  & Personnelle \\
            \textbf{Durée}   & 3 (1 / round) \\
        \end{tabular}
        \\ \\
        \'Elément basique dans la formation des Jedi ou des Sith, la vitesse de Force est un pouvoir permettant à l’utilisateur d’atteindre des vitesses élevées sur des courtes distances. Suivant le potentiel de l’utilisateur, la vitesse ou la durée de la vélocité variera. Cette technique permet d’atteindre des lieux plus rapidement. Une vitesse très élevée permettrait même au Jedi de percevoir son environnement au ralenti et ainsi d’acquérir une plus grande précision au combat.

        Vitesse de Force permet à la cible de se déplacer plus rapidement. Avec un succès, l’Allure est doublée. Avec une Relance, courir devient une action gratuite, n’infligeant donc plus le malus de -2 habituel. 
        \\

    \item [Poussée de Force] ~ \\

        \begin{tabular}{ r l }
            \textbf{Rang}    & Aguerri \\
            \textbf{Points}  & 3 \\
            \textbf{Portée}  & 3m ou 6m \\
            \textbf{Durée}   & 1 action \\
        \end{tabular}
        \\ \\
        Le principe de ce pouvoir est simple : l’utilisateur d’un tel pouvoir crée par simple concentration un champ de Force qui, même s’il n’est pas visible, est tout de même bien réel et capable de repousser un adversaire. Utilisé généralement dans une seule direction, la poussée de Force peut néanmoins être utilisée comme une onde qui se déplace de façon concentrique tout autour de l’utilisateur, repoussant tout objet ou être vivant sur son passage. 

        Toute personne se trouvant dans la zone d’effet doit faire un jet de Force (avec un malus de -2 en cas de Relance). En cas d’échec, la victime est projetée en arrière de 2d6 cases et se retrouve au sol. Si la cible heurte un objet inanimé, elle est en outre Secouée. Les cibles bénéficiant d’un abri peuvent soustraire la protection de couverture au nombre de cases parcourues (avec un minimum de 0) et les créatures volantes subissent un malus de -2 au jet de Force.
        \\

\end{description}

\subsubsection{Pouvoirs \'Elémentaires}
\begin{description}[align=left] 

    \item [Dissimulation de Force] ~ \\

        \begin{tabular}{ r l }
            \textbf{Rang}    & Aguerri \\
            \textbf{Points}  & 2 \\
            \textbf{Portée}  & Visuelle \\
            \textbf{Durée}   & 3 (1 / round) ou 1 heure (1 / heure) \\
        \end{tabular}
        \\ \\
		La dissimulation de Force permet aux utilisateurs de la Force suffisamment talentueux de dissimuler leur affiliation à la Force (à savoir s’ils sont du Côté Lumineux ou Obscur), leur capacité à la contrôler et même parfois d’occulter leur présence par rapport à d’autres êtres sensibles à la Force. 

		Si un personnage utilise Sens de Force pour déceler un autre personnage qui utilise Dissimulation, leur Jet d’Arcane (Force) s’oppose.
        \\

	\item [Persuasion de Force] ~ \\

        \begin{tabular}{ r l }
            \textbf{Rang}    & Vétéran \\
            \textbf{Points}  & 3 \\
            \textbf{Portée}  & Touché \\
            \textbf{Durée}   & 1 action \\
        \end{tabular}
        \\ \\
		La Persuasion de Force permet à son utilisateur de contrôler l’esprit d’un être organique. Elle est aussi appelée Domination de Force par les Jedi et les Sith qui la maîtrisent très bien. Pour reprendre les propos d’Obi-Wan Kenobi : "Grâce à la force, on peut influencer les esprits faibles". 

		Le jet d’Arcane (Force) est opposé au jet d’\^Ame de la cible. En cas de succès, la victime fera ce qui lui est ordonné sans poser de question. En cas d’échec critique la victime est consciente que le personnage essaye de la manipuler grâce à la Force.
        \\

    \item [Télékinésie] ~ \\

        \begin{tabular}{ r l }
            \textbf{Rang}    & Aguerri \\
            \textbf{Points}  & 5 \\
            \textbf{Portée}  & \^Ame \\
            \textbf{Durée}   & 3 (1 / round) \\
        \end{tabular}
        \\ \\
        La Télékinésie est le pouvoir de la Force le plus basique qui soit et l’un des premiers à être enseigné, après peut-être la façon de sentir la Force autour de soi. Il s’agit de l’art de déplacer des corps matériels simplement en faisant mentalement appel à la Force.

        Un utilisateur de la Force peut soulever cinq fois sont dé d’\^Ame avec un succès et 25 fois son dé d’\^Ame avec une Relance.

        Les cibles vivantes peuvent résister à ce pouvoir avec un jet d’\^Ame opposé au jet d’Arcane. En cas de succès, la cible n’est pas affectée. En cas d’échec par contre, elle est soulevée comme l’indique ce pouvoir et ne bénéficie pas d’autres chances d’y échapper. Une victime peut occasionnellement réussir à s’agripper à quelque chose d’assez solide pour essayer de résister à Télékinésie. Dans ce cas, il doit réussir un jet de Force opposé à un jet d’Arcane de l’arcaniste. En cas de succès, il a réussi à s’agripper et ne peut pas être déplacé ou écrasé ce round.
        \\

    \item [Télépathie] ~ \\

        \begin{tabular}{ r l }
            \textbf{Rang}    & Aguerri \\
            \textbf{Points}  & 3 \\
            \textbf{Portée}  & Infinie \\
            \textbf{Durée}   & 3 (1 / round) \\
        \end{tabular}
        \\ \\
        La Télépathie est la capacité à pouvoir communiquer mentalement avec d’autres individus proches ou éloignés par une importante distance. Cette capacité n’impliquant pas d’autre interaction sensorielle ou énergétique, les échanges entre les deux personnes sont donc virtuellement inviolables. Pour pouvoir utiliser ce pouvoir, certaines conditions doivent être respectées. Tout d’abord, le télépathe doit avoir une affinité avec la Force. 

        La cible doit réussir un jet d'\^Ame pour recevoir le message. Pour répondre, la cible doit posséder le même pouvoir et l’utiliser de la même façon en retour mais avec un bonus de +2 au jet d’Arcane. La réponse est automatiquement reçue.

        Pour une discussion complexe, on compte 3 PP pour les 3 premiers aller/retour, au delà pour chaque aller/retour, l’initiateur de la discussion consomme 1 PP supplémentaire.
        \\

    \item [Voile de Force] ~ \\

        \begin{tabular}{ r l }
            \textbf{Rang}    & Vétéran \\
            \textbf{Points}  & 5 \\
            \textbf{Portée}  & Personnelle \\
            \textbf{Durée}   & 3 (1 / round) \\
        \end{tabular}
        \\ \\
        Le Voile de Force, appelé plus communément le camouflage de Force, est une capacité de la Force aux applications très pratiques, surtout si l’on désire passer inaperçu. Bien que le principe de cette aptitude soit assez simple à comprendre, il consiste pour un utilisateur de la Force à détourner les rayons lumineux qui lui arrivent dessus afin d’empêcher qu’ils ne soient réfléchis et donc que la personne soit visible ; arriver à la maîtriser demande un niveau très élevé de maîtrise de la Force. Peu sont ceux qui peuvent prétendre arriver à utiliser le camouflage de Force. Néanmoins, comme toute technique, elle a ses limites.

        Déjà, pour peu qu’on y prête vraiment attention, il est possible pour un observateur attentif de distinguer les courbes du corps de l’utilisateur de ce pouvoir. En effet, assurer la continuité visuelle entre l’environnement et la personne dissimulée est très difficile sur les contours du corps. De plus, on ne peut pas se fier à ce pouvoir pour passer inaperçu lorsqu’on se retrouve face à des droïdes ou des races utilisant un système de visualisation dont le spectre lumineux est différent ou plus performant que celui d’un humain. Une vision basée sur les infrarouges, ou sur d’autres longueurs d’onde que celles du spectre lumineux humain ne sera pas trompée par le subterfuge.

        Avec un succès, le personnage est transparent mais sa silhouette est visible. Il est possible de le détecter en ayant une raison de se douter de sa présence et en réussissant un jet de Perception avec un malus de -2. Avec une Relance le malus passe à -4. Dans les deux cas, le pouvoir affecte le personnage et les objets qu’il transporte.
        \\

\end{description}

\subsubsection{Pouvoirs Universels}

Les Pouvoirs Universels sont des pouvoirs génériques pouvant être rattachés à n’importe quelle philosophie sans distinction. Il arrive que certaines personnes considèrent les pouvoirs élémentaires comme un sous groupe des Pouvoirs Universels. Néanmoins, la distinction peut être opérée sur les pouvoirs qu’on enseigne généralement au Padawan. En effet, les Pouvoirs Universels sont moins enseignés que les pouvoirs élémentaires, ce qui les rend plus rares et moins connus. 

\begin{description}[align=left] 

    \item [Déflexion de Force] ~ \\

        \begin{tabular}{ r l }
            \textbf{Rang}    & Aguerri \\
            \textbf{Points}  & 2 \\
            \textbf{Portée}  & Toucher \\
            \textbf{Durée}   & 3 (1 / round) \\
        \end{tabular}
        \\ \\
        La Déflexion de Force est un pouvoir autant utilisé par les Jedi que par les Sith, tous deux y recourant lorsqu’ils souhaitent dévier des attaques et lorsqu’ils ne disposent pas d’un sabre laser ou ne désirent pas y recourir.

        Avec un succès, les attaquants reçoivent un malus de -2 à leurs jets de Combat, Tir et autres jets d’attaque contre l’utilisateur. Une Relance augmente ce malus à -4. Ce pouvoir fonctionne comme Armure contre les armes à aire d’effet.
        \\

    \item [Méditation de Combat] ~ \\

        \begin{tabular}{ r l }
            \textbf{Rang}    & Aguerri \\
            \textbf{Points}  & 2 \\
            \textbf{Portée}  & Infinie \\
            \textbf{Durée}   & 3 (1 / round) \\
        \end{tabular}
        \\ \\
        La Méditation de combat est une manifestation particulière de la Force dont le but est de galvaniser les troupes en leur donnant du courage, alors que chez les troupes ennemies ce pouvoir permet de réduire leur volonté de combattre pratiquement à néant. Ce pouvoir, dont la sphère d’influence est immense et requiert donc une concentration absolue, s’avère très utile pour gagner des batailles, parfois en ne versant que peu de sang.

        Ce pouvoir confère à son utilisateur l’atout \emph{Meneur d’hommes} pour ces alliés. S’il est dirigé contre des adversaires, ces derniers utilisent aussi un d10 comme dés joker mais c’est le plus mauvais score qui est conservé et pas le meilleur.
        \\

    \item [Téléportation] ~ \\

        \begin{tabular}{ r l }
            \textbf{Rang}    & Vétéran \\
            \textbf{Points}  & 3 \\
            \textbf{Portée}  & 10m \\
            \textbf{Durée}   & Instantané \\
        \end{tabular}
        \\ \\
        La Téléportation est un pouvoir très rare permettant à son utilisateur de mouvoir son corps instantanément. Son action est comparable à celle de la Télékinésie mais sur soi-même. L’utilisateur du pouvoir détermine mentalement son point de destination et appelle la Force à le faire disparaître puis réapparaître à cette position. La distance qu’il est possible de parcourir n’excède pas les quelques mètres. Ce pouvoir peut aussi prendre la forme d’une lévitation.

        Ce pouvoir compte pour le déplacement du round. Les adversaires adjacents au personnage n’ont pas d’attaque gratuite contre lui.
        \\

\end{description}

\newpage
\subsubsection{Pouvoirs du Côté Clair}

Les Pouvoirs du Côté Clair sont identifiables par leurs actions bénéfiques ou défensives. Mais ces facultés peuvent avoir également des effets sur l’individu lui-même, comme une amélioration de ses facultés physiques ou mentales. A noter également que certains de ces pouvoirs peuvent être employés dans un cadre offensif, mais jamais pour nuire directement. 

\textbf{Contrecoup}: En cas d’échec critique, le héros est automatiquement Secoué.

\begin{description}[align=left] 

    \item [Chaleur de Force] ~ \\

        \begin{tabular}{ r l }
            \textbf{Rang}    & Vétéran \\
            \textbf{Points}  & 2 \\
            \textbf{Portée}  & Toucher \\
            \textbf{Durée}   & 3 (1 / round) \\
        \end{tabular}
        \\ \\
        C’est un pouvoir du Coté Lumineux. Grâce à lui, un Jedi peut insuffler un élan de Force dans son propre corps ou dans celui d’un allié. Ce qui a pour effet d’augmenter la combativité de la cible. En outre, ce pouvoir peut aussi être utilisé sur un adversaire pour l’affaiblir.

        Ce pouvoir permet à un personnage d’augmenter la Force de la cible d’un type de dé en cas de succès et de deux en cas de Relance. Le Trait affecté peut dépasser d12 : chaque augmentation au-delà octroie un bonus de +1 au Trait. Par exemple, une Relance obtenue sur une cible ayant déjà d12 dans le Trait affecté donnerait un Trait à d12+2 pour la durée du pouvoir. 

        Le pouvoir peut également être utilisé pour réduire le Trait d’un adversaire. Il s’agit alors d’un jet d’Arcane opposé à l’Âme de la cible. En cas de Succès, le Trait est réduit d’un type de dé et de deux en cas de Relance. Un Trait ne peut être réduit en dessous de d4. Il est possible de cumuler les effets de ce pouvoir, mais l’arcaniste devra noter séparément les durées.
        \\

    \item [Guérison de Force] ~ \\

        \begin{tabular}{ r l }
            \textbf{Rang}    & Vétéran \\
            \textbf{Points}  & 3-10-20 \\
            \textbf{Portée}  & Toucher \\
            \textbf{Durée}   & 3 (1 / round) \\
        \end{tabular}
        \\ \\
        Lorsqu’un Jedi est blessé lors d’un combat, il peut utiliser une technique nommée Guérison de Force qui lui permet de recouvrer ses forces, de panser ses blessures en accélérant le processus naturel de guérison en utilisant la Force. Lorsqu’il est peu familier avec ce pouvoir, celui qui veut l’utiliser doit se plonger dans une transe afin de mettre en \oe{uvre} cette technique. Toutefois, plus la maîtrise de ce pouvoir grandit, moins l’utilisateur a besoin de se couper du monde et de sombrer dans une transe Jedi pour panser ses plaies, ces dernières se guérissent d’elles-mêmes rapidement. Mais l’efficacité est toujours plus grande si le Jedi se plonge en transe. Outre le fait de guérir des plaies superficielles, la guérison de Force peut reconstituer des os et faire repousser de la chair. 

        Pour 3 points de pouvoir, chaque utilisation de Guérison efface une blessure avec un succès et deux avec une Relance. Le jet de dé subit une pénalité égale au nombre de blessures de la victime (en plus de tout malus dont le lanceur de sort lui-même serait affublé). Cela ne fonctionne que pendant la première heure après la blessure. Le pouvoir de Guérison peut également servir à guérir les poisons et les maladies, s’il est lancé moins de 10 minutes après l’événement déclencheur.

        Pour 10 Points de pouvoir, il peut également être utilisé pour neutraliser un poison ou guérir une maladie au-delà des 10 premières minutes. Grande guérison peut également soigner les blessures permanentes et incapacitantes. Le jet d’Arcane se fait alors à -4 et le nombre de Points de pouvoirs nécessaires est de 20. Une seule tentative est possible pour une telle blessure. En cas d’échec de ce pouvoir, la blessure est définitivement permanente.
        \\

    \item [Lumière de la Force] ~ \\

        \begin{tabular}{ r l }
            \textbf{Rang}    & Vétéran \\
            \textbf{Points}  & 3 \\
            \textbf{Portée}  & Toucher \\
            \textbf{Durée}   & 30 minutes (1 / 10 minutes) ou 3 (1 / round) \\
        \end{tabular}
        \\ \\
        La Lumière de la Force est un pouvoir du Côté Lumineux assez impressionnant et bénéfique puisque, comme son nom l’indique, il apporte la lumière dans les lieux les plus obscurs de la galaxie. En d’autres termes, la Lumière de la Force permet à un utilisateur du Côté Lumineux de venir à bout d’une manifestation du Côté Obscur. Et ce genre de manifestation obscure peut être soit un esprit, une personne ou un lieu, appelé alors nexus, c’est-à-dire un endroit de concentration d’énergie mystique, d’énergie maléfique en l’occurrence.

        Le jet d’Arcane est opposé à un jet d’\^Ame de la cible. En cas de succès, la cible est Secouée. En cas de Relance, elle est hors de combat. Pour les Jokers en cas de relance la cible subit une blessure.

        Ce pouvoir, en plus de blesser une cible, annule jusqu’à 6 points de malus dus à l’obscurité dans un rayon de 6m.
        \\

    \item [Protection de Force] ~ \\

        \begin{tabular}{ r l }
            \textbf{Rang}    & Vétéran \\
            \textbf{Points}  & 2 \\
            \textbf{Portée}  & Toucher \\
            \textbf{Durée}   & 3 (1 / round) \\
        \end{tabular}
        \\ \\
        La Protection de Force est un pouvoir du Côté Lumineux qui confère à son utilisateur une meilleure protection aux agressions de contact. Cet un pouvoir très ancien.

        Les protections apportées par ce pouvoir sont diverses : absorption des projectiles laser, atténuation des blessures causées par une lame ou par une explosion, absorption des attaques de Force.

        Lorsqu’il est utilisé comme protection pendant un combat, un succès donne 2 points d’Armure et une Relance 4 points.
        \\

    \item [Adaptation de Force] ~ \\

        \begin{tabular}{ r l }
            \textbf{Rang}    & Vétéran \\
            \textbf{Points}  & 2 \\
            \textbf{Portée}  & Toucher \\
            \textbf{Durée}   & 1 heure (1 / heure) \\
        \end{tabular}
        \\ \\
        Ce pouvoir permet de résister aux conditions environnementales censées être invivables pour les êtres organiques. Ainsi un Jedi peut traverser une plaine rocailleuse sous une pluie acide battante pendant plus d’une heure et s’en sortir indemne.

        Ce pouvoir permet de respirer, parler et de se déplacer à son Allure normale sous l’eau, dans des environnements sans gravité, dans le vide, la lave d’un volcan, des steppes arctiques, etc\ldots La pression, l’atmosphère, l’air ainsi que tous les aspects nécessaires à la vie du personnage sont fournis par ce pouvoir. La protection est complète, mais ne concerne que l’environnement. Les dégâts causé par un combat, une embuscade ou un évènement extérieur seront encaissé par le héros.

        Cibles supplémentaires : l’utilisateur peut affecter jusqu’à cinq cibles en repayant 2 PP par cible supplémentaire.
        \\

\end{description}

\subsubsection{Pouvoirs du Côté Obscur}

A l’opposé des Pouvoirs du Côté Clair se trouvent ceux du Côté Obscur, considérés comme étant des pouvoirs maléfiques. Leur objectif n’est pas de protéger ou soigner mais de nuire ou détruire à l’image de l’\'Eclair de Force ou de l’\'Etranglement de Force. A la différence des pouvoirs du Côté Clair, l’utilisation prolongée de certains pouvoirs du Côté Obscur peut altérer le corps de l’utilisateur.

\textbf{Contrecoup}: Chaque Succès dans l’utilisation d’un de ces pouvoirs diminue le Charisme du héros de -1, -2 en cas de Relance. En cas d’échec critique l’utilisateur est automatiquement Secoué mais ne subit pas de malus de Charisme.

\newpage
\begin{description}[align=left] 

    \item [\'Eclair de Force] ~ \\

        \begin{tabular}{ r l }
            \textbf{Rang}    & Vétéran \\
            \textbf{Points}  & 1-3 ou 2 \\
            \textbf{Portée}  & 12 \\
            \textbf{Durée}   & Instantanée \\
        \end{tabular}
        \\ \\
        Pouvoir emblématique du Côté Obscur de la Force, l’Éclair de Force est une arme à usage purement offensif et dont le but principal n’est rien moins que tuer son adversaire. Prenant la forme de multiples éclairs d’énergie pure que l’on pourrait assimiler à de l’électricité, les éclairs de Force peuvent être dévastateurs selon le niveau de maîtrise de l’attaquant. 

        Les dégâts d’un Éclair sont de 2d6. \\
        
        \textbf{Éclairs supplémentaires} : le personnage peut lancer jusqu’à 3 Éclairs au prix d’un Point de Pouvoir supplémentaire par Éclair. Décidez du nombre d’Éclairs créés avant d’activer le pouvoir. Les Éclairs peuvent être répartis sur des cibles différentes au choix du Sith. Ce dernier fait un jet comme pour un tir avec une arme automatique mais sans le malus de tir automatique : le Sith fait un jet de Compétence d’Arcanes pour chaque éclair, qu’il oppose à la Difficulté pour toucher chaque cible.\\
        
        \textbf{Dégâts supplémentaires} : le personnage peut également porter les dégâts d’un Éclair à 3d6 en dépensant 2 Points de pouvoir. Il ne peut pas lancer d’éclairs supplémentaires lorsqu’il utilise cette option.
        \\

    \item [\'Etranglement de Force] ~ \\

        \begin{tabular}{ r l }
            \textbf{Rang}    & Vétéran \\
            \textbf{Points}  & 3 \\
            \textbf{Portée}  & 12 \\
            \textbf{Durée}   & Instantanée \\
        \end{tabular}
        \\ \\
        Associé au Côté Obscur de la Force, l’Étranglement de Force est sans doute l’un des procédés les plus cruels pour tuer une cible. Dérivée de l’étreinte de Force, cette technique est identique à un étranglement classique, à ceci près qu’elle applique une forte pression sur la gorge d’un individu grâce à la télékinésie. A la différence d’un autre pouvoir de la Force, celui-ci est tout à fait spécifique car il se focalise sur une seule partie du corps de l’individu, en l’occurrence le cou. 

        Ce pouvoir inflige un jet d’Arcane (Force) par round tant qu’il est maintenu et la cible ne peut faire d’autre action que de tenter de se libérer. Chaque tour le Jet d’Arcane (Force) est opposé à un jet d’\^Ame de la cible.

        Un allié de la cible peut tenter un jet d’\^Ame en opposition au jet d’Arcane (Force) du Sith pour libérer la victime. En cas de succès l’allié perturbe le Sith et ce dernier aura un malus de -2 à son prochain jet d’Arcane (Force), en cas de relance le Sith lâche prise immédiatement.

        Si le Sith est secoué, il subit un malus de -2 à son prochain jet d’arcane (Force), si le Sith est blessé il subit un malus de -4.

        Tant que le pouvoir est maintenu, le Sith ne peut attaquer une autre cible et il subit un malus de -2 en Parade.
        \\

    \item [Drain de Force] ~ \\

        \begin{tabular}{ r l }
            \textbf{Rang}    & Vétéran \\
            \textbf{Points}  & 3 \\
            \textbf{Portée}  & Visuelle \\
            \textbf{Durée}   & Instantanée \\
        \end{tabular}
        \\ \\
        Le pouvoir appelé Drain de Force permet à un adepte du Coté Obscur de recharger sa puissance de Force.

        L’arcaniste sélectionne une cible dans la portée et fait un jet d’Arcane (Force) opposé. Avec un succès, il draine 1d6+1 PP à sa victime et 1d8+2 avec une Relance. Ces jets ne font pas d’As. La cible ne peut passer en dessous de 0 PP. Le pouvoir ne fonctionne que sur les créatures ayant l’Atout d’Arcane (Force).
        \\

    \item [Restructuration Mentale] ~ \\

        \begin{tabular}{ r l }
            \textbf{Rang}    & Vétéran \\
            \textbf{Points}  & 3 \\
            \textbf{Portée}  & Toucher \\
            \textbf{Durée}   & 3 (1 / round) \\
        \end{tabular}
        \\ \\
        Ce pouvoir de la Force (ou plutôt du Coté Obscur) est sûrement le plus puissant en ce qui concerne la manipulation mentale. En effet, il surpasse de loin la Persuasion de Force par exemple. On ne sait pas vraiment comment ce pouvoir fonctionne, mais on peut facilement imaginer que l’utilisateur va dans un premier temps briser la volonté de la cible (on ne sait pas si cette dernière, par une volonté de fer, peut y résister). Une fois que l’individu est sans défense, sa personnalité est brisée pour ensuite être remodelée (l’utilisateur se prenant évidement pour modèle) selon la volonté du "manipulateur". 

        Restructuration Mentale utilise un jet d’Arcane (Force) opposé à un jet d’\^Ame de la cible. En cas de succès, la victime attaquera ses alliés et ira même jusqu’à se donner la mort, bien qu’un tel acte octroie un nouveau jet d’\^Ame pour échapper à l’emprise.
        \\

    \newpage
    \item [Tempête de Force] ~ \\

        \begin{tabular}{ r l }
            \textbf{Rang}    & Légendaire \\
            \textbf{Points}  & 10 \\
            \textbf{Portée}  & \^Ame \\
            \textbf{Durée}   & Instantané \\
        \end{tabular}
        \\ \\
        La Tempête de Force est un pouvoir que seuls les utilisateurs du Côté Obscur arrivent à maîtriser et est probablement la plus puissante aptitude de la Force connue à ce jour. En fait, l’appellation de tempête de Force est erronée, même si c’est l’impression que ce pouvoir donne. 

        L’utilisateur de ce pouvoir crée un vortex de Force capable d’infligé de lourds dégâts aux personnages à portée. Le vortex peut être créé n’importe ou même dans l’espace mais pas au delà du système planétaire où se trouve l’utilisateur. La portée du vortex est égale à l’\^Ame du Sith. Ceux qui se trouvent à portée subissent 2d10 de dégâts.

        En cas d’échec critique c’est l’utilisateur qui subit ces dégâts. Ce pouvoir est considéré comme une arme lourde.
        \\

\end{description}
